\documentclass[a4paper,12pt]{urmia-report}
\usepackage{urmia-style}

\begin{document}
	
	% ------------------ صفحه عنوان ------------------

    \thispagestyle{empty}  % بدون شماره صفحه و بدون هدر/فوتر

	\begin{center}
	    \textbf{ به نام خدا}\\[1.5cm]
		
		\includegraphics[width=0.28\textwidth]{logo.png}\\[1cm]
		
		\textbf{دانشکده برق ، کامپیوتر و فناوری های پیشرفته}\\[1cm]
		
		\textbf{گروه مهندسی کامپیوتر}\\[1.5cm]
		
		{\Large \textbf{تمرین درس.....}}\\[1.5cm]
		
		{\Large \textbf{عنوان تمرین }}\\[1.5cm]
		
		
			\textbf{نام و نام خانوادگی:}  \\[0.6cm]
			\textbf{شماره دانشجویی:}  \\[0.6cm]
			\textbf{استاد:}  \\[0.6cm]
			
		
		\vfill
		\today \\[0.3cm]
	    \textbf{تمامی حقوق این اثر متعلق به دانشگاه ارومیه می باشد.}
	\end{center}
	
	\newpage

	\setcounter{page}{1}    
	\pagenumbering{arabic}  
	
	
	        %- ---------------------- چکیده ----------------------------

	\newsection{چکیده}
	
	در این بخش چکیده تمرین نوشته شود ....
	\newpage
	%-------------------------فهرست مطالب ----------------------
	\tableofcontents
	\newpage
	
	%- ---------------------- مقدمه ----------------------------
    \newsection{مقدمه}
    در این بخش مقدمه تمرین نوشته شود ....
	
	\newpage
	
	
	%- ----------------- کد پیاده‌سازی شده و توضیحات آن --------------------
	\newsection{کد پیاده‌سازی شده و توضیحات آن}
	در صورت داشتن کد ، توضیحات مربوط به آن در این بخش نوشته شود....
	
	\newpage
	
	
	%- ----------------- خروجی نسخه اول --------------------
	\newsection{خروجی نسخه اول }
	در این بخش ، خروجی اولیه کد نوشته شود ....
	
	\newpage
	
	% -------------------- مشکل نسخه اول و رفع آن --------------------
	\newsection{مشکل نسخه اول و رفع آن }
	
	مشکلات نسخه اولیه کد (در صورت وجود) در این بخش توضیح داده شود....
	
	
	\newpage
	
	% -------------------- نسخه بهبودیافته و توضیح آن --------------------
	\newsection{نسخه بهبودیافته کد و توضیح آن }
	
	نسخه بهبودیافته کد و توضیحات مربوط به آن (در صورت وجود) در این بخش نوشته شود ....
	\newpage
	
	
	% --------------------مزایای کد بهبودیافته --------------------
	\newsection{مزایای کد بهبودیافته }
	در این بخش به مزیت های کد بهبودیافته اشاره کنید ....
	
	\newpage
	
	
	% --------------------خروجی کد بهبودیافته --------------------
	\newsection{خروجی کد بهبودیافته }
	
	خروجی نسخه کد بهبودیافته در این بخش نوشته شود ....
	\newpage
	
	
	% --------------------تحلیل نمودار ها --------------------
	\newsection{تحلیل نمودار ها }
	تحلیل و توضیحات مربوط به نمودار ها(در صورت وجود) در این بخش نوشته شود ....
	
	\newpage
	
	
	% --------------------مقایسه نتایج دو نسخه --------------------
	\newsection{مقایسه نتایج دو نسخه }
	در صورت نیاز به مقایسه میتوانید در این بخش توضیحات را ارائه دهید....
	\newpage
	
	
	% -------------------- پاسخ به سوالات کلیدی --------------------
	\newsection{پاسخ به سوالات کلیدی }
	پاسخ به سوالات کلیدی تمرین در این بخش نوشته شود ....
	\newpage
	
	
	% -------------------- جمع بندی --------------------
	\newsection{جمع بندی }
	جمع بندی مربوط به تمرین در این بخش نوشته شود ....
	
	\newpage
	
	
	% -------------------- نتیجه‌گیری --------------------
	\newsection{ نتیجه‌گیری}
	
	نتیجه گیری مربوط به تمرین در این بخش نوشته شود ....
	
	\newpage
	
	
	% -------------------- فهرست منابع --------------------
    \newsection{ فهرست منابع}
    \begin{sources}
    	
    	\item
    	\href{}
    	{\nolinkurl{}}
    	
    	\item
    	\href{}
    	{\nolinkurl{}}
    	
    	\item
    	\href{}
    	{\nolinkurl{}}
    	
    \end{sources}
    
    
\end{document}
